Prezentujemy implementację Prinza - rozszerzenia do oprogramowania Nussknacker przeznaczonego
do zarządzania i ewaluacji modeli uczenia maszynowego w tymże środowisku. W pracy opisujemy
potrzebę powstania naszej biblioteki i tematykę związaną z przetwarzaniem dużej liczby danych
zdarzeniowych. Przedstawiamy nasz proces tworzenia biblioteki open source, włączając w to
specyfikację API projektu oraz jego implementacje w postaci integracji z MLflow, standardem
PMML oraz H2O. Ponadto opisujemy, jak przebiegał proces prowadzenia projektu open source,
przygotowania wygodnego środowiska dla programisty i decyzje podjęte podczas implementacji
poszczególnych integracji w ramach przyjętego interfejsu. Prezentujemy nasze konkluzje
dotyczące pracy z poszczególnymi integracjami i procesem projektowania ich wdrożenia do
naszego środowiska, jak i dodatkowe założenia powstałe podczas procesu implementacji.
Jako rezultat naszych prac dostarczamy źródła naszej biblioteki wraz z czytelnym przykładem
jej jej wykorzystania w środowisku Nussknackera.
