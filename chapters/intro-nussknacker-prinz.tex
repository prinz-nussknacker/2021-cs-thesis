\chapter{Introduction to Nussknacker and Prinz}
\label{chap:intro-nussknacker-prinz}

\section{Nussknacker}

Nussknacker is an event-stream processing and decision-making solution developed by TouK, a software house based in Warsaw.
It's a fully open-source project available on GitHub.
The project has been in development since late 2016, and at the moment of writing this thesis it has almost 200 stars, 35 contributors, and a codebase of almost 180 thousand lines.
Nussknacker lets the user design, deploy, and monitor streaming processes through an easy-to-use GUI.
It's intended as a simple way for non-programmers to write and customize processes.
The user creates a diagram describing the flow of data from many types of blocks available in the Nussknacker UI (e.g. filters or aggregators).
The described process can then be tested and run on an Apache Flink cluster.

\subsection{Use cases}

Historically, the initial use case for Nussknacker was Real-Time Marketing or RTM.
One of TouK's clients had some large data streams, which they intended to use for their marketing campaigns.
The key here was the ability to process and manipulate the data quickly.
However, most modern stream processing engines require the user to know a domain-specific programming language.
Therefore, Nussknacker allowed non-technical users - like analysts or managers to process large amounts of data and draw actionable insights.

Nowadays, the other main use case for Nussknacker is fraud detection, in particular in the telecom business.
When dealing with some kinds of fraud (e.g. SMS spamming) it's necessary to take instant, automated action.
Changes to those actions shouldn't each time require additional development.
It's especially important for companies that might not have an internal programming team.
In such a case, analysts and other users with little or no programming background can design and monitor the processes themselves using Nussknacker.

\section{Prinz}

Prinz is a library of extensions for Nussknacker.
It provides a simple API, that allows developers to add new integrations with machine learning engines or repositories.
At the moment integrations with 3 tools - MLFlow, PMML, and H2O are available in Prinz.
Prinz integrations are highly configurable - each can provide its way of storing the models and retrieving data from them.
Each integration includes one or more model repositories that are used for different model accessing strategies.
When integration is added to Nussknacker, each model listed in those repositories becomes available in the Nussknacker UI as a block.
This block can then be integrated into the normal flow of the designed process and for example used as a filter or an aggregator.
