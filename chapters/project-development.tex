\chapter{Project development}
\label{chap:project-development}

This chapter describes approaches to managing work and developing the project.

\section{Project management}

\subsection{Stakeholders}

There were three parties involved in the project.

\paragraph{Development team.}
Authors of this thesis worked on the project for its whole duration.
Students had little or no prior knowledge of Scala or machine learning solutions.
Maciej Procyk took the role of a team leader on a day-to-day basis.

\paragraph{Touk sp. z.o.o. s.k.a.}
Touk\footnote{\href{https://touk.pl/}{https://touk.pl/}} is a software house which ordered the integrations in response to a rising demand for machine learning solutions in Nussknacker.
Maciej Próchniak was the contact person and a technical consultant for the project.

\paragraph{University of Warsaw.}
Authors developed the project for a bachelor thesis at The Faculty of Mathematics, Informatics and Mechanics of the University of Warsaw.
Janusz Jabłonowski was the contact person for legal issues.
Grzegorz Grudziński supervised the project itself.

\subsection{Communications}

Prinz was developed during the Covid-19 pandemic, which influenced the usual development workflow.
Except for actions required by law, all meetings were held online.

Team developed the code asynchronously, meeting every weekly for a sprint retrospective and discussing new issues.
They used Slack to bridge a gap arising from lack of face-to-face consultations during the week.
Each Thursday delegated members met with the supervisor to discuss development progress.
Team also held Friday meetings with the commissioning party to encourage bidirectional flow of information.

\subsection{Timeline}

The project started in October 2020 with an estimated deadline of June 2021.
The list below presents important milestones.
\begin{enumerate}
  \item \textbf{October 1, 2020}: Start of development.
  \item \textbf{January 7, 2021}: Presentation of different approaches to serving machine learning solutions at university seminar. Project will later include integrations for these technologies.
  \item \textbf{January 27, 2021}: MLflow integration merged to \texttt{master}. Start of the PMML subproject.
  \item \textbf{February 4, 2021}: Demo of the MLflow integration to the client.
  \item \textbf{March 21, 2021}: Start of the H2O subproject.
  \item \textbf{April 15, 2021}: PMML integration merged to the main branch.
\end{enumerate}

\section{Development}

\subsection{Tools and technologies}

Prinz is a set of extensions for Nussknacker.
It adopts similar concepts as the parent project to ease integration and developer onboarding.

Core parts of the solution are Scala 2.12 code built with the \texttt{sbt} tool.
External libraries often supply only a Java API, which requires occasional use of Java classes.
Example models (along with some adapter code for the serving endpoints) were prepared in Python 3 due to its popularity among the data science community \cite{srinath2017python}.

Team uses \texttt{.git} as the project's control version system.
Files are hosted remotely on GitHub\footnote{\href{https://github.com/}{https://github.com/}}, which also provides tools to manage issues, pull requests and project boards.
While there are no requirements regarding developer tools, all members opted for using IntelliJ IDEA, an IDE often used by Scala engineers.
Documentation and specifications use Markdown where possible.

\subsection{Programming workflow}

Development processes in the project rely heavily on the features provided by GitHub.

For each new issue, engineer creates a development branch. Branch names follow naming convention \texttt{username/feature-name}.
Developer uses this branch as a development area.
Once the feature is ready, they create a pull request to the main project branch.

Pull request is a final stage before merging changes to the code base.
Its creation triggers several automated checks (called ''GitHub workflows'') to enforce code quality and reveal regressions.
Code quality evaluation includes simple checks (e.g. for extra white space) and static analysis tools like \texttt{scalastyle}.
Regression testing comprises unit tests of the project.

Other team members can review the code and request changes.
Developer usually pushes new commits with the updates (what in turn triggers workflows again).
They mark conversation as resolved and ask team members for approval.
With an approval and all checks passing, developer checks-in their code to the main branch.

Team introduced \texttt{dev/feature} branches at a later stage of the project.
These are dedicated for complex changes involving multiple contributors.
Pushing to \texttt{dev/feature} branches follows the same workflow.

\subsection{Interesting issues}

\paragraph{Protected branches.}
To prevent accidental changes, \texttt{master} branch was marked as protected.
This means that contributors can't push to \texttt{master} directly.
Also, pull requests require an approval and passing checks and cannot be merged otherwise.
The same rules apply to \texttt{dev/feature} branches.

\paragraph{Additional documentation.}
Project includes additional documentation served as a GitBook\footnote{\href{https://gitbook.com/}{https://gitbook.com/}} web page.
Manual pages are written in Markdown and stored along the code.
This approach allows to automate building and deploying HTML pages using existing solutions.
Choice of GitBook and its version follows on from the parent project.
