\chapter{Project development}
\label{chap:project-development}

\todo[inline]{TODO(kantoniak): Revisit in June.}

\section{Project management}

\todo[inline,caption={}]{
  TODO:
  \begin{itemize}
	\item For Touk: answering to a need, open-source project
	\item Timeline
	\item Weekly meetings
	\item Issues and boards
  \end{itemize}
}

\section{Development}

\subsection{Tools and technologies}

Prinz is a set of extensions for Nussknacker.
It adopts similar concepts as the parent project to ease integration and developer onboarding.

Core parts of the code base are written in Scala 2.12 and built with the \texttt{sbt} tool.
External libraries often provide only an API for Java, which requires occasional use of Java classes.
Example models (along with some adapter code for the serving endpoints) were prepared in Python 3 due to its popularity among the data science community \cite{srinath2017python}.

Team uses \texttt{.git} as the project's control version system.
Files are hosted remotely on GitHub\footnote{\href{https://github.com/}{https://github.com/}}, which also provides tools to manage issues, pull requests and project boards.
While there are no requirements regarding developer tools, all members opted for using IntelliJ IDEA, an IDE often used by Scala engineers.
Documentation and specifications use Markdown where possible.

\subsection{Programming workflow}

Development processes in the project rely heavily on the features provided by GitHub.

For each new issue, engineer creates a development branch. Branch names follow naming convention \texttt{username/feature-name}.
Developer uses this branch as a development area.
Once the feature is implemented, they create a pull request to the main project branch.

Pull request is a final stage before merging changes to the code base.
Its creation triggers a number of automated checks (called ''GitHub workflows'') to enforce code quality and reveal regressions.
Code quality evaluation includes simple checks (e.g. for extra white space) and static analysis tools like \texttt{scalastyle}.
Regression testing comprises unit tests of the project.

Other team members can review the code and request changes.
Developer usually pushes new commits with the updates (what in turn triggers workflows again).
They mark conversation as resolved and ask team members for approval.
With an approval and all checks passing, developer checks-in their code to the main branch.

Team introduced \texttt{dev/feature} branches at a later stage of the project.
These are dedicated for complex changes involving multiple contributors.
Pushing to \texttt{dev/feature} branches follows the same workflow.

\subsection{Interesting issues}

\paragraph{Protected branches.}
To prevent accidental changes, \texttt{master} branch was marked as protected.
This means that contributors can't push to \texttt{master} directly.
Also, pull requests require an approval and passing checks and cannot be merged otherwise.
The same rules apply to \texttt{dev/feature} branches.

\paragraph{Additional documentation.}
Project includes additional documentation served as a GitBook\footnote{\href{https://gitbook.com/}{https://gitbook.com/}} web page.
Manual pages are written in Markdown and stored along the code.
This approach allows to automate building and deploying HTML pages using existing solutions.
Choice of GitBook and its version follows on from the parent project.
