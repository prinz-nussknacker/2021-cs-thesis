\chapter{H2O integration}
\label{chap:h2o}

\section{Introduction}

H2O is one of the leading machine learning platforms.
It features an open-source server with in-memory implementations of multiple popular algorithms.
Users can access the server through hosted visual notebooks or using multiple programming languages, including R and Python.
AutoML functionality helps to automatically select matching model types for provided datasets.
Created models can be deployed as
POJOs\footnote{POJO -- \emph{Plain Old Java Object}}
or MOJOs\footnote{MOJO -- \emph{Model Object, Optimized} -- is an alternative storage standard by H2O.ai.}
for scoring.

H2O.ai, the company behind H2O, created the platform to democratize access to AI solutions.
Its multiple products are fully open source, while the company provides paid enterprise support.
With over 12000 customers in different industries, H2O.ai was classified as a Visionary in Gartner 2021 Quadrant for Data Science and Machine Learning, surpassing companies like Google and Microsoft in terms of \emph{Completeness of vision}.

\section{H2O in Prinz environment}

H2O provides both a server endpoint and a set of Java libraries for automated scoring based on deployment files.
We chose to load files directly and score them in memory by libraries, mainly because of an open license (Apache 2.0.) and compatibility between Java and Scala languages.
To score incoming data, our implementation performs the following steps:
\begin{enumerate}
  \item load model, from memory, file or URL,
  \item map Nussknacker types to H2O types (\texttt{String} or \texttt{Double}),
  \item convert incoming data to rows,
  \item score rows individually,
  \item extract output values and retype outputs for Nussknacker,
  \item gather results.
\end{enumerate}
This approach creates a few challenges, described in the following sections.

\section{Loading models}

When creating a model, H2O outputs POJO or MOJO files which can be downloaded by the user.
The scoring library needs to know the file type and how to acquire its data.
Since Nussknacker UI displays all exposed models, it needs to know about them when creating enrichers in the configuration phase.

We chose MOJO as a supported type.
POJO seems to be a predecessor format with limitations on file size and worse performance.
While the company will support POJO files, MOJO is the recommended file format \cite{h2o-mojo-over-pojo}.

As for locating data, there is no standard for discovering models.
One approach could be to download models directly from the H2O server, but this defeats the purpose of using small, portable files.
This also prevents infrastructure setups without an exposed server.

For simplicity and coherence, we opted for a custom server implementation based on the same interface as in PMML integration.
Any repository client implements methods for listing and accessing files.
Built-in implementations allow for using files from a local directory or loading a list of URLs from a remote endpoint.
For details on PMML implementation, see section \ref{sec:pmml-repository}.

\section{Extracting signatures}

General properties of the model (like name or version) can be provided by the repository as in the case of the PMML repository.
All metadata about the model contents is described inside deployment files.
Because of this, we have to load all file models when loading configuration, even if a particular model won't be used.
On the flip side, H2O libraries have full knowledge of the model and can provide names and types of inputs and outputs without any further analysis.

\section{Scoring models}

At this point, Prinz does not support entry batching, but internal API operates on sets of rows.
H2O libraries can score only single rows, so sets have to be ungrouped into single rows, scored, and then grouped back.
The input data for scoring models has to be converted before evaluating the H2O model instance. This step is needed as
in the whole flow of data inside Nussknacker, the \texttt{}{String} values don't have extra quotes while the H2O needs to
have the non-numerical input data enclosed in extra quotes.

The result of scoring is a \emph{prediction}, with exact representation depending on type of the model.
H2O supports some other types of algorithms beyond traditional ML models.
Predictions have to been transformed to extract output values for Nussknacker.
Currently supported types are:
\begin{itemize}
  \item \texttt{AnomalyDetection},
  \item \texttt{Binomial},
  \item \texttt{Clustering},
  \item \texttt{CoxPH},
  \item \texttt{KLime},
  \item \texttt{Multinomial},
  \item \texttt{Ordinal},
  \item \texttt{Regression}.
\end{itemize}
